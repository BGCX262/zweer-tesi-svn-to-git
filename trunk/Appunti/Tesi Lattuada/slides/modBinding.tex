\begin{slide}{Binding sul tipo di unita' funzionale}
\begin{itemize}
\item E' stata aggiunta la funzione di binding di un'operazione su un tipo di unit� funzionale;
\item si associa una forza ad \red ogni terna <operazione-passo di controllo-tipo di unit� funzionale> \black;
\item $FORCE(o,s,t) = self\_force(o,s,t) + pred\&succ\_force(o,s,t) + \red assign\_force(o,s,t) \black$
\item al concetto di probabilit� di un'operazione di essere schedulata in un passo di controllo si sostituisce quello di \red percentuale di occupazione di un tipo di unit� funzionale in un passo di controllo da parte di un'operazione \black;
\item in caso di mancanza di binding la nuova formulazione coincide con quella originale; 
\item � possibile utilizzare questa formulazione per tutte le casistiche del problema di scheduling.
\end{itemize}
\end{slide}
