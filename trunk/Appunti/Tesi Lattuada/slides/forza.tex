\begin{slide}{Formula per il calcolo della forza}
\begin{itemize}
\item Il termine forza deriva dall'analogia con la legge di Hooke: \[F = -kx\]
\item ad ogni coppia <operazione-passo di controllo> � assegnata una probabilit�: $prob(o,s)=\frac{1}{mobilita'}$
\item $FORCE(o,s) = self\_force(o,s) + pred\&succ\_force(o,s)$
\item $self\_force(o,s) = force(o,s)$
\item $pred\&succ\_force(o,s) = \sum_{q\epsilon pred(o)} force(q,s) + \sum_{q\epsilon succ(o)} force(q,s)$
\item $force(o,s) = \sum_{p\epsilon wind(o)} contrib(o,p)$
\item $contrib(o,p) = DG(type(o), p)\cdot \triangle prob(o,p)$
\end{itemize}
\end{slide}
