\chapter*{Ringraziamenti}


Il primo ringraziamento e il pi� doveroso � per i miei genitori che hanno sopportato la mia parte studentesca (c'� una parte studentesca in tutti) per ben diciannove anni ed � un grosso merito perch� durante i periodi di studio-compiti-esami credo di essere stato abbastanza insopportabile. Con il loro supporto mi hanno dimostrato che molteplici e opposti possono essere i modi per spingere una persona verso il suo obiettivo. Voglio poi dedicare un ricordo ai miei nonni perch� sicuramente � anche merito loro se sono diventato quello che sono.

Ringrazio poi il Prof. Fabrizio Ferrandi: non solo per il supporto e l'aiuto datomi durante il lavoro di tesi, ma soprattutto perch� � grazie agli insegnamenti dei suoi corsi se all'interno del mare magnum dell'informatica mi sono appassionato alla progettazione di sistemi dedicati ed ho quindi continuato il percorso che mi ha portato a questa tesi di laurea. Ironicamente devo anche ringraziare chi involontariamente mi ha spinto a provare a percorrere questa strada imponendo certe scelte nei piani di studio.

Estendo quindi il ringraziamento a tutti quei professori, ricercatori e dottorandi che ho incontrato durante questi sei anni di studio al Poli: a ciascuno di loro devo un pezzetto pi� o meno grande di questa laurea e di quella del primo livello. A questo punto dovrei ringraziare anche i miei colleghi con cui ho fatto qualche progetto ma lo far� in seguito perch� sono prima amici che colleghi.

Ringrazio quindi tutti quelli con cui ho percorso un tratto pi� o meno breve di questi anni, a partire dagli amici con cui ho condiviso i primi semestri e di cui ormai a distanza di cinque anni ricordo pi� il soprannome che il nome: il mio ``quasi parente'' Edo (chiss� se sei poi andato in Scandinavia), Alain (spero si scriva cos�), Matteo, Francesco, Silvia, il capozarro, Pocchio e altri che probabilmente in questo momento dimentico. Un particolare ringraziamento a Lucky con cui ho condiviso oltre alla vita dentro e fuori l'universit� la tragica esperienza del pendolarismo ferroviario; il pensare che sono ormai undici anni che ci conosciamo mi fa rendere conto di quanto siamo ormai ``vecchi'' (tu resti comunque pi� vecchio di ben una settimana). Ecco, a proposito di pendolarismo, NON ringrazio le Ferrovie dello Stato per tutte le ore perse in questi anni, i tentativi andati a vuoto di farmi mancare un esame e per i numerosi tentativi di farmi ammalare tramite caldo, freddo e mancanza di ossigeno. E' stata una dura prova di sopravvivenza, ma sono riuscito a uscirne abbastanza indenne (ma rimane ancora il giorno della laurea da affrontare).

Ringrazio quindi tutti gli amici-colleghi con cui ho condiviso il percorso di questa Laurea Specialistica all'interno del percorso degli esami ``HW-SW'', fra cui soprattutto Dario anche per l'esperienza del lavoro svolto insieme, e Miele che � stato molto di pi� che un semplice compagno di progetto e gli amici del Laboratorio di Ing. Software.

Ho voluto lasciare per ultimi i ragazzi con cui maggiormente ho condiviso questa grande esperienza che � la vita universitaria (che � ben diverso dal condividere l'universit�!) e cio� i ragazzi della mailingleet (facciamo un po' di pubblicit� che non fa mai male \href{www.leet.it}{www.leet.it}; pubblicit� a cosa? andate a vederlo). Grazie a loro per avermi fatto vivere e contribuire a formare la filosofia del \emph{lost} e del \emph{leet}. ``Losto'' la sua descrizione perch� ci vorrebbe una tesi solo per questo. Fra i tanti del gruppo in particolare voglio ricordare il Mazzu e la sua pazzia matematica (temo sia definitivamente irrecuperabile), il Lazza e il suo TAF, il Bongo (sono io o sei tu che si deve preoccupare del fatto che appena ho pensato al Lazza ho pensato anche a te?), quella fonte infinita di lost che � Giani, il Plinio che ci ha sempre mostrato come un ingegnere informatico debba vivere, Carlo che invece ci ha sempre mostrato come un ing. debba realizzare un progetto, gli interisti (Maio, Claudio e il Berti) con cui ho discusso di un calcio che si � poi rivelato falso, Geo con cui condivido la fede calcistica (Forza Milan!!), il professor ``Lale'' e poi Domenico, CareLuca, Teo Parabiago, il Ninja, Michele, Aladino (lo so che sembra strano, ma non � un soprannome) e tutti gli altri iscritti alla mailing.

Dimentico qualcuno? Probabilmente s� perch� in questi giorni ormai sono abbastanza fuso. Se ci� dovesse essere accaduto me ne scuso con gli interessati. Grazie a tutti!


