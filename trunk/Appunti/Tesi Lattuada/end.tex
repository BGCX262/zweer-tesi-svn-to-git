\chapter{Conclusioni e possibili sviluppi futuri}
\label{c:end}
\thispagestyle{empty}

\vspace{0.5cm}

In questo lavoro di tesi � stato proposto un possibile algoritmo di scheduling basato sul Force Directed che consente di cercare di minimizzare il numero di unit� funzionali allocate in un sistema estraendo il massimo parallelismo possibile dalla specifica del sistema descritto attraverso un SDG. A differenza degli algoritmi tradizionali, il metodo proposto permette di considerare contemporaneamente vincoli temporali e vincoli relativi alle risorse. Questa completezza � pagata in termini di tempi di computazione che rispetto ad altri algoritmi pi� semplici sono notevolmente maggiori. Tuttavia considerando i guadagni non trascurabili in termini di area e conseguentemente in termini di potenza dissipata si pu� essere disposti a pagare questo tempo in fase di progettazione del sistema.

I risultati dei test proposti hanno dimostrato la bont� delle modifiche effettuate e l'utilit� nell'aver creato un algoritmo che gestisca contemporaneamente vincoli temporali e sulle risorse cercando di minimizzare una metrica relativa al costo dell'implementazione.

La versione dell'algoritmo presentata si � quindi dimostrata sufficientemente completa e potrebbe anche essere considerata come definitiva. Esistono tuttavia ancora dei possibili margini di miglioramento e delle possibili estensioni. Infatti ci sono ancora degli aspetti secondari rimasti insoluti all'interno del lavoro presentato che potrebbero costituire l'oggetto di futuri lavori:
\begin{itemize}
\item individuare un criterio pi� accurato per assegnare la priorit� alle operazioni, in particolare determinando il criterio per l'ordinamento totale dei blocchi basici che fornisca i migliori risultati in termini di unit� funzionali allocate e tempo di computazione, ed eventualmente introdurre la possibilit� di assegnare priorit� diverse alle operazioni appartenenti ad uno stesso blocco basico (cio� schedulare un blocco basico a pezzi invece che interamente), in modo tale da ridurre il tempo di computazione anche in problemi privi di costrutti condizionali;

\item estendere l'algoritmo modificato per minimizzare altri aspetti del costo relativo all'implementazione quali numero di registri o costo delle interconnessioni;

\item individuare come inserire all'interno dell'algoritmo informazioni relative al diverso costo delle unit� funzionali in presenza di vincoli; in questo caso infatti l'algoritmo che � stato presentato cerca di minimizzare il numero complessivo di unit� funzionali appartenenti a tipi non vincolati indistintamente; sicuramente per� si vorrebbe che la minimizzazione del numero di risorse non vincolate di tipo pi� costoso fosse favorita rispetto a quella delle unit� pi� economiche; l'algoritmo originale veniva indirizzato in questo senso attraverso pesi, calcolati sulla base del costo delle unit� funzionali, attribuiti alle diverse somme di probabilit�; � evidente che lo stesso metodo deve essere applicato anche nella versione proposta, ma sussiste il problema di come combinare i pesi derivanti dal costo delle unit� funzionali con quelli relativi ai vincoli;

\item valutare se vi sia un effettivo miglioramento nella qualit� dei risultati introducendo un calcolo meno approssimativo delle \emph{predecessors' and successors' forces} che tenga conto delle dipendenze indirette fra le riduzioni delle mobilit� di operazioni contemporanee che sono predecessori o successori dell'operazione di cui si sta esaminando l'assegnamento;

\item soprattutto da un punto di vista implementativo, considerare la possibilit� di ridurre il tempo di computazione dell'algoritmo, nel caso di presenza di vincoli, a scapito della memoria da esso utilizzata: tenendo memorizzati i dati relativi ad ogni nodo esaminato nell'esplorazione dell'albero di ricerca � possibile ridurre il numero di percentuali di occupazione o di forze da ricalcolare in caso di backtracking.
\end{itemize}
