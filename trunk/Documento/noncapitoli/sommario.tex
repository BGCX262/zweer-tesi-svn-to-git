\begin{sommario}

Al giorno d'oggi la sempre pi� vasta domanda di contenuti audiovisivi specifici sta obbligando i vari distributori di servizi quali \corsivo{Video on Demand} e \corsivo{IP Television} a utilizzare algoritmi di raccomandazione per aiutare l'utente nella fase di scelta.

Nel corso degli anni � stata creata una gran quantit� di algoritmi in grado di effettuare una raccomandazione e quest'enorme offerta ha reso la vita incredibilmente difficile a tutti coloro che han dovuto tenere aggiornate le varie piattaforme di suggerimenti e test sull'efficacia degli stessi. Ed � proprio questa fase di test che pi� preoccupa in quanto � computazionalmente molto impegnativa. 

La ricerca della \corsivo{semplificazione} e dell'\corsivo{estendibilit�} � l'obiettivo che ci si pone in questa tesi: per un momento si vuole interrompere la ricerca di nuovi algoritmi o nuove metodologie di test, in favore di una reingegnerizzazione di tutto il sistema di raccomandazione, rendendolo pi� leggibile anche ai non addetti ai lavori e soprattutto facilmente estendibile per i nuovi algoritmi che verranno sicuramente ideati negli anni a venire.

Il tutto viene realizzato appoggiandosi a quell'incredibile strumento che � Matlab, in modo da tralasciare l'ideazione di strutture dati complesse, lasciando tutta la complessit� al linguaggio di programmazione. Matlab, infatti, implementa delle funzioni in grado di calcolare in maniera molto rapida complesse strutture matematiche, che altrimenti richiederebbero una fase di programmazione molto pi� lunga.

\end{sommario}