% Comando per il corsivo
\newcommand{\corsivo}[1]{\textit{#1}}

% Comando per il grassetto
\newcommand{\grassetto}[1]{\textbf{#1}}

% Comando per il sottolineato
\newcommand{\sottolineato}[1]{\underline{#1}}

% Comando per la nota
\newcommand{\nota}[1]{\footnote{#1}}

% Comando per la nota della bozza
\newcommand{\notabozza}[1]{\nota{\grassetto{Nota Bozza:} #1}}

% Comando per gli apici
\newcommand{\apice}[1]{\textsuperscript{#1}}

% Comando per i pedici
\makeatletter 
  \DeclareRobustCommand*\textsubscript[1]{% 
    \@textsubscript{\selectfont#1}} 
  \newcommand{\@textsubscript}[1]{% 
    {\m@th\ensuremath{_{\mbox{\fontsize\sf@size\z@#1}}}}} 
\makeatother 
\newcommand{\pedice}[1]{\textsubscript{#1}}

% Comando per i quote da usare come note
\newcommand{\quotenota}[1]{\begin{initialquote}\grassetto{Nota di bozza:} #1\end{initialquote}}

% Stile Matlab per LstListing
\lstdefinestyle{Matlab}{language=Matlab}