\chapter{Conclusioni}
\label{c:conc}

La reingegnerizzazione dell'applicazione Matlab per i sistemi di raccomandazione � stata una tappa obbligata nello sviluppo della piattaforma.

I benefici sono evidenti:

\begin{description}
	\item[Miglioramento nella gestione della memoria] in quanto prima dell'intervento le matrici venivano passate alle varie funzioni e memorizzate in modo completo. Ora invece si cerca di utilizzare un sistema simile ai puntatori di alcuni linguaggi di programmazione.
	\item[Riduzione dello spazio in memoria] poich� sono stati eliminati un incredibile numero di file inutili che erano stati creati solo come prove o fini a se stessi. A dimostrazione di ci� si � passati da una memoria sul filesystem di oltre $ 4.5 Gb $ a poco pi� di $ 31 Mb $\nota{Gran parte dello spazio richiesto dall'applicazione prima di questo lavoro era occupato da enormi matrici \urm e \icm. Nella nuova piattaforma non sono state incluse, ma al loro posto sono state inserite matrici molto pi� piccole, ma sempre esemplificative}.
	\item[Informazioni pi� accessibili] dal momento che i vari risultati vengono memorizzati all'interno degli oggetti che li hanno prodotti. In questo modo non � pi� necessario, da parte dell'utente, utilizzare numerose variabili poich� ne � sufficiente una per ogni oggetto istanziato.
	\item[Espandibilit�] per quanto riguarda i nuovi algoritmi e le nuove metodologie di test. Grazie all'utilizzo di apposite API � possibile implementare nuove funzioni che gli oggetti vanno a recuperare in modo automatico, senza che l'utente debba tutte le volte specificare il percorso dei file che gli interessano.
	\item[Facilit� d'uso] dal momento che ci si basa su un esiguo numero di comandi con i quali si riesce a compiere praticamente qualsiasi tipo di operazione. Star� poi all'utente esperto disporre dei parametri come meglio crede per ottenere un diverso livello di interazione con l'algoritmo.
	\item[Esportazione] per poter visualizzare i risultati graficamente o importarli in altri sistemi in modo, ad esempio, di valutare pi� facilmente certe statistiche non implementate in questa sede.
\end{description}

\section{Sviluppi futuri}
\label{c:conc:svi}

Come epilogo di questo lavoro vengono inserite delle possibilit� di sviluppo che si sono presentate durante lo studio dell'applicazione. Non tutte sono prioritarie, anzi, alcune possono definirsi quasi inutili, ma per dovere di cronaca vengono qui menzionate:

\begin{itemize}
	\item Realizzazione di un'interfaccia grafica per consentire l'uso dell'applicazione non solo da riga di comando. Questa potrebbe aprire la piattaforma, ma anche i sistemi di raccomandazione in generale, a un pubblico sempre pi� vasto, ma soprattutto non prettamente di addetti ai lavori. Un'interfaccia grafica poi permetterebbe anche di avere un'idea pi� precisa di quello che si sta realizzando, sia per quanto riguarda le liste di suggerimenti, sia per quanto riguarda i risultati dei test.
	\item Esportare i risultati dei test in grafici, in modo da visualizzarne meglio l'andamento. In questo modo si vuole offrire un'alternativa all'esportazione testuale che � gi� presente, ma quasi inutile in considerazione della mole di risultati. Inoltre la codifica \csv adottata mal si presta a rappresentare enormi array di valori in quanto bisogna creare troppe convenzioni per i separatori dei dati.
	\item Una migliore implementazione degli algoritmi, in modo da considerare le matrici a colonne e non pi� a righe. Questo porterebbe a un incremento notevole nelle performance: l'ideale sarebbe implementare tutte le funzioni in C e quindi costruire in Matlab solamente ""l'interfaccia''. Parallelamente a questo lavoro, per�, non sono stati condotti studi che ne verificassero l'effettivo miglioramento anche per operazioni brevi.
\end{itemize}